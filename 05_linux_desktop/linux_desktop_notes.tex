\documentclass[slides]{pgnotes}

\title{Linux desktop}

\begin{document}

\maketitle

\tableofcontents

\section{Remote desktop protocol}

\textbf{Remote Desktop Protocol (RDP)} is commonly used to access Windows desktops:
\begin{itemize}
\item RDP runs on \textbf{Port 3389}.
  \begin{itemize}
  \item Normally TCP but can use UDP as well for better performance.
  \end{itemize}
\item Clients on many different OSes.
  \begin{itemize}
  \item RDP client is normally standard on Windows client installations.
  \item Usually available, even in restricted client environments.
  \end{itemize}
\item Normally a session can only be accessed by a single RDP or physical console session.
  \begin{itemize}
  \item Remote assist breaks this rule!
  \item Usually the latest connection ``picks up'' a running session.
  \end{itemize}
\item On Windows a session can transfer between physical console and RDP session.
\end{itemize}


\subsection{Terminal services}

\begin{center}
  \includegraphics[width=0.8\linewidth]{terminal_services}
\end{center}



\subsection{Linux}

Linux can provide terminal-services-like functionality:
\begin{itemize}
\item Historically linux has used an alternative protocol called VNC to access graphical desktops remotely.
\item Also has remote-X for single applications (only good on LAN!)
\item Alternative is XRDP which provides an RDP-compatible server:
  \begin{itemize}
  \item Standard Windows Remote Desktop Connection client can be used.
  \item Great in restricted client environments.
  \end{itemize}
\item Depending on setup can do either:
  \begin{itemize}
  \item Floating desktop per user (volatile or persistent).
  \item Attach to physical console.
  \end{itemize}
\end{itemize}

\subsection{RDP server}

\begin{center}
  \includegraphics[width=0.7\linewidth]{rdp}
\end{center}



\section{Scenario}



\subsection{Motivation}



\end{document}
